% $Id: examples.tex,v 1.5 2001-06-15 07:38:53 geuzaine Exp $

%
% WARNING:
%
% If \bigpictures is set to 1, the pictures must have been checked out
% (cvs module name is getdp-picts) in the ../../../getdp-picts directory.
%
% ---------------------------------------------------------------------------

\begin{slide}

\slidepagestyle{none}

\begin{center}
\bigtitle{Examples}\\
\ifnum\fulltitle=1\par\bigskip\bigskip
\mediumtitle{Patrick Dular and Christophe Geuzaine}\\
\bigskip
\smalltitle{Department of Electrical Engineering}\\
\smalltitle{Montefiore Institute B28, Sart Tilman Campus}\\
\smalltitle{University of Li�ge}\\
\smalltitle{B-4000 Li�ge (BELGIUM)}
\fi
\end{center}

\end{slide}

% ---------------------------------------------------------------------------
\part{GetDP/Gmsh examples}
% ---------------------------------------------------------------------------

\chapter{Magnetostatics}

\ifnum\bigpictures=1
\ifx\pdfoutput\undefined

\begin{slide}

\begin{center}
\includegraphics[width=0.38\textwidth]{getdp-picts/ind1}%
\includegraphics[width=0.38\textwidth]{getdp-picts/ind2}

\includegraphics[width=0.38\textwidth]{getdp-picts/ind3}%
\includegraphics[width=0.38\textwidth]{getdp-picts/ind4}
\end{center}

\end{slide}

\fi
\fi

\begin{slide}

\mybox{colbox}{\textwidth}{
\begin{equation*}
\Curl{\vec{h}} = \vec{j} ,\quad
\Div{\vec{b}} = 0 \quad\text{and}\quad
\vec{b} = \mu \vec{h} + \mu_0 \vec{h}_m 
\end{equation*}
\begin{equation*}
\begin{split}
\xymatrix{
 \color{colpos}\phi    \ar@{->}[r]^-{\GradSymb_h}  &
 \vec{h} \ar@{->}[r]^-{\CurlSymb_h} \ar@{<->}[d]^{\mu} &
 \vec{j} \ar@{->}[r]^-{\DivSymb_h}   &
 0 \\
 0       \ar@{<-}[r]^-{\DivSymb_e}&
 \vec{b} \ar@{<-}[r]^-{\CurlSymb_e}&
 \color{colpos}\vec{a} 
}
\end{split}
\end{equation*}
}

\begin{slideitemize}
\item Weak form of Gauss law: 
\begin{equation*}
%\ivol{\Div{\vec{b}}}{\phi'} = 0 \Rightarrow
\ivol{\vec{b}}{\Grad{\phi'}} + \isur{\psca{\vec{n}}{\vec{b}}}{\phi'} 
= 0
\quad \forall\phi'\in\Hone[_0]{\Omega}
\end{equation*}

\item Weak form of Ampere's law:
\begin{equation*}
%\ivol{\Curl{\vec{h}}}{\vec{a}'} = \ivol{\vec{j}}{\vec{a}'} \Rightarrow
\ivol{\vec{h}}{\Curl{\vec{a}'}} + \isur{\pvec{\vec{n}}{\vec{h}}}{\vec{a}'}
= \ivol{\vec{j}}{\vec{a}'} 
\quad \forall\vec{a}'\in\Hcurl[_0]{\Omega}
\end{equation*}

\end{slideitemize}

\end{slide}

% ---------------------------------------------------------------------------

\chapter{Magnetostatics: $\vec{b}$-conform}

\begin{slide}

\mybox{colbox}{\textwidth}{
\begin{center}
\emph{Vector potential} formulation
\begin{equation*}
\vec{b} = \Curl{\vec{a}}
\end{equation*}
\end{center}
}

\begin{center}
$\Downarrow$\\
Weak form of Amp�re's law\\
$\Downarrow$\\
$\displaystyle\ivol{\mu^{-1}\Curl{\vec{a}}}{\Curl{\vec{a}'}} 
= \ivol{\vec{j}}{\vec{a}'} ,
\quad\forall\vec{a}'\in\Hcurl[_0]{\Omega}$
\end{center}

\bigskip
NB: gauge for $\vec{a}$, ...

\end{slide}

\ifnum\bigpictures=1
\ifx\pdfoutput\undefined
\background{7\semcm}{0\semcm}{\includegraphics[width=0.38\textwidth]{getdp-picts/ind1}}
\fi\fi

\begin{slide}

\begin{smallsyntax}
Group \{
  Core = 1; Inductor = 2; SkinInductor = 3, Air = 4;
  Omega = Region[\{Core, Inductor, Air\}];
\}
Function \{
  mu0 = 4.e-7 * Pi; mur = 1000;
  mu[ Core ] = mur * mu0;
  mu[ Region[\{Air, Inductor\}] ] = mu0;
  j[ Inductor ] = ...; //to be defined
\}
Constraint \{
  \{ Name a; 
    Case \{ 
      \{ Region CL_a0; Value 0; \}
    \}
  \}
\}
\end{smallsyntax}

\end{slide}

\background{}{}{}

\begin{slide}

\begin{smallsyntax}
FunctionSpace \{
  \{ Name Hcurl; Type Form1; \CC{vector potential}
    BasisFunction \{
      \{ Name se;  NameOfCoef ae; Function BF_Edge; Support Omega; 
        Entity EdgesOf[All]; \} \CC{associated with the edges of the mesh}
    \}
    Constraint \{ \CC{essential constraint + gauge (unicity)}
      \{ NameOfCoef ae;  EntityType EdgesOf ; NameOfConstraint a; \}
      \{ NameOfCoef ae;  EntityType EdgesOfTreeIn ; EntitySubType StartingOn;
        NameOfConstraint Gauge ; \}
    \}
  \}
\}
\end{smallsyntax}

\end{slide}

\begin{slide}

\begin{smallsyntax}
Formulation \{
  \{ Name MagSta_a; Type FemEquation;
    Quantity \{ 
      \{ Name a ; Type Local; NameOfSpace Hcurl; \}
    \}
    Equation \{
      Galerkin \{ [ 1/mu[] * Dof\{Curl a\} , \{Curl a\} ]; 
                 In Omega; Integration I1; Jacobian JVol; \}
      Galerkin \{ [ -j[] , \{a\} ]; 
                 In Inductor; Integration I1; Jacobian JVol; \}
    \}
  \}
\}
\end{smallsyntax}

\end{slide}

\begin{slide}

\begin{smallsyntax}
Resolution \{
  \{ Name MagSta_a;
    System \{
      \{ Name A; NameOfFormulation MagSta_a; \}
    \}
    Operation \{ Generate A; Solve A; SaveSolution A; \}
  \}
\}
PostProcessing \{
  \{ Name test; NameOfFormulation MagSta_a;
    Quantity \{
      \{ Name a; Value \{ Local\{ [ \{a\} ]; In Omega; \} \} \}
      \{ Name normb; Value \{ Local\{ [ Norm[\{d a\}] ]; Omega; \} \} \}
    \}
  \}
\}
\end{smallsyntax}

\end{slide}

\begin{slide}

Magnetodynamics?

Additional term in the formulation: 

\begin{smallsyntax}
      Galerkin \{ DtDof [ sigma[] * Dof\{a\} , \{a\} ]; 
                 In Core; Integration I1; Jacobian JVol; \}   
\end{smallsyntax}

New resolution:
\begin{smallsyntax}
  \{ Name MagDyn_a_t; \CC{time domain}
    System \{
      \{ Name A; NameOfFormulation MagDyn_a; \}
    \}
    Operation \{ 
      InitSolution[A]
      TimeLoopTheta[0,20/50,0.1/50,1]\{
        Generate[A]; Solve[A]; SaveSolution[A];
      \}
    \}
  \}
\end{smallsyntax}


\end{slide}

% ---------------------------------------------------------------------------

\chapter{Magnetostatics: $\vec{h}$-conform}

\begin{slide}

\mybox{colbox}{\textwidth}{
\begin{center}
\emph{Magnetic field conform} formulation
\begin{equation*}
\vec{h} = \vec{h}_s+\vec{h}_r ,\quad\text{with}\quad
\Curl{\vec{h}_s} = \vec{j}     \quad\text{and}\quad
\vec{h}_r = -\Grad{\phi}
\end{equation*}
\end{center}
}

\begin{center}
$\Downarrow$\\
Weak form of Gauss law\\
$\Downarrow$\\
$\displaystyle\ivol{\mu(-\Grad{\phi}+\vec{h}_s)}{\Grad{\phi'}} 
= 0 ,
\quad \forall\phi'\in\Hone[_0]{\Omega}$
\end{center}

\bigskip
NB: choice of source field $\vec{h}_s$, treatment of multiply connected
$\Omega$, ...

\end{slide}

\begin{slide}

\begin{smallsyntax}
FunctionSpace \{
  \{ Name H1; Type Form0; \CC{scalar potential}
    BasisFunction \{
      \{ Name sn; NameOfCoef phin; Function BF_Node; Support Omega; 
        Entity NodesOf[All]; \} \CC{associated with the nodes of Omega}
    \}
    Constraint \{ \CC{essential constraint}
      \{ NameOfCoef phin; EntityType NodesOf; NameOfConstraint phi; \}
    \}
  \}
\}
\end{smallsyntax}

\end{slide}

\begin{slide}

\begin{smallsyntax}
Formulation \{
  \{ Name MagSta_phi; Type FemEquation;
    Quantity \{
      \{ Name phi; Type Local; NameOfSpace H1; \}
      \{ Name hs; Type Local; NameOfSpace Hcurl_s; \}  \CC{a little patience...}
    \}
    Equation \{
      Galerkin \{ [ mu[] * \{hs\} , \{Grad phi\} ];
                 In Omega; Integration I1; Jacobian JVol;  \}
      Galerkin \{ [ mu[] * Dof\{Grad phi\} , \{Grad phi\} ]; 
                 In Omega; Integration I1; Jacobian JVol;  \}
    \}
  \}
\}

\end{smallsyntax}

\end{slide}

\begin{slide}

\begin{smallsyntax}
FunctionSpace \{
  \{ Name Hcurl_s; Type Form1; \CC{space for the source field}
    BasisFunction \{
      \{ Name se; NameOfCoef he; Function BF_Edge; Support Inductor; 
        Entity EdgesOf[All, Not SkinInductor]; \}
      \{ Name sc; NameOfCoef Ic; Function BF_GradGroupOfNodes; 
        Support Transition; Entity GroupsOfNodesOf[Cut]; \}
      \{ Name sc; NameOfCoef Icc; Function BF_GroupOfEdges; 
        Support Inductor; Entity ...; \}
    \}
    Constraint \{
      \{ NameOfCoef he; EntityType EdgesOfTreeIn; EntitySubType StartingOn;
        NameOfConstraint Gauge; \}
      \{ NameOfCoef Ic;  EntityType GroupsOfNodesOf; NameOfConstraint I; \}
      \{ NameOfCoef Icc; EntityType GroupsOfNodesOf; NameOfConstraint I; \}
    \}
  \}
\}
\end{smallsyntax}

%... -> GroupsOfEdgesOf[Cut, InSupport ElementsOf[SkinInductor, OnOneSideOf Cut] ]

\end{slide}

% ---------------------------------------------------------------------------

\chapter{Magneto-thermal coupling}

\begin{slide}
....
\end{slide}

% ---------------------------------------------------------------------------

\background{}{}{}

\ifnum\bigpictures=1
\ifx\pdfoutput\undefined

\chapter{Other examples...}

\begin{slide}

\begin{center}
\includegraphics[width=\textwidth]{getdp-picts/antenna1}
\includegraphics[width=\textwidth]{getdp-picts/antenna3}
\includegraphics[width=\textwidth]{getdp-picts/antenna2}
\includegraphics[height=\textheight]{getdp-picts/indheat}
\includegraphics[angle=-90,width=\textwidth]{getdp-picts/line220kv}
\includegraphics[height=\textheight]{getdp-picts/magnet}
\includegraphics[height=\textheight]{getdp-picts/motoras}
\includegraphics[width=\textwidth]{getdp-picts/f16a}
\includegraphics[width=\textwidth]{getdp-picts/f16b}

\hspace*{-0.2\textwidth}%
\includegraphics[width=0.5\textwidth]{getdp-picts/p20induc2}\hspace*{-0.2\textwidth}
\includegraphics[width=0.5\textwidth]{getdp-picts/p20}\hspace*{-0.2\textwidth}%
\includegraphics[width=0.5\textwidth]{getdp-picts/p20ada}\hspace*{-0.2\textwidth}%

piezo-electricity, magnetostriction, non-homogeneous waveguides, photonic
cristals, electromagnetic shielding, dielectric heating, ...

\end{center}



\end{slide}

\else

\fi
\fi
