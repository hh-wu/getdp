% $Id: gmsh.tex,v 1.9 2001-06-15 07:38:53 geuzaine Exp $

% ---------------------------------------------------------------------------

\begin{slide}

\slidepagestyle{none}

\begin{center}
\bigtitle{Gmsh --- Geometry, mesh, solver integration
          and visualization}\\
\ifnum\fulltitle=1\par\bigskip\bigskip
\begin{minipage}{0.5\textwidth}\center
\mediumtitle{Christophe Geuzaine}\\
\bigskip
\smalltitle{Dept. of Electrical Engineering}\\
\smalltitle{Montefiore Institute B28, Sart Tilman}\\
\smalltitle{University of Li�ge}\\
\smalltitle{B-4000 Li�ge (BELGIUM)}
\end{minipage}%
\begin{minipage}{0.5\textwidth}\center
\mediumtitle{Jean-Fran�ois Remacle}\\
\bigskip
\smalltitle{Scientific Computation Research Center}\\
\smalltitle{Rensselaer Polytechnic Insitute}\\
\smalltitle{CII 110 8th Street}\\
\smalltitle{Troy, New York 12180-3590 (USA)}
\end{minipage}
\fi
\end{center}

\end{slide}

% ---------------------------------------------------------------------------
\part{Gmsh}
% ---------------------------------------------------------------------------

\chapter{Finite element methods in practice}

\begin{slide}

Four main steps:
\begin{slideitemize}
\item \emph{Geometry}: CAD description
\item \emph{Mesh}: structured or unstructed mesh generator
\item \emph{Solver}: GetDP :-)
\item \emph{Post-processing}: visualization (2D and 3D)
\end{slideitemize}

Most of the time necessary to solve a problem is \emph{not} spent in the
solver step!

\end{slide}

% ---------------------------------------------------------------------------

\chapter{Geometry description}

\begin{slide}

Bottom-to-top definition:
\begin{slideitemize}
\item \emph{Points}
\item Oriented \emph{curves} (segments, circles, ellipsis, splines, etc.)
\item Oriented \emph{surfaces} (plane, ruled, etc.)
\item \emph{Volumes}
\end{slideitemize}

Parametrization:
\begin{slideitemize}
\item Dedicated \emph{language}
\item \emph{Scripting} possibilities (loops, tests, arrays of variables,
etc.)
\end{slideitemize}

%For complex models, a top-to-bottom approach (Constructive Solid Geometry)
%is better.

\end{slide}

% ---------------------------------------------------------------------------

\chapter{Mesh generation}

\begin{slide}

\begin{slideitemize}
\item \emph{Structured} (transfinite, elliptic, hyperbolic)
\item \emph{Unstructured} (Delaunay triangles/tetrahedra)
\begin{enumerate}
\item 
mesh of a box including the convex polygon/polyhedron resulting from
curves/surfaces discretization
\item
initial mesh by insertion of curves/surfaces nodes by the Bowyer algorithm
\item 
boundary restoration to force presence of all edges/faces
\item 
suppression of undesired triangles/tetrahedra
\item
new node insertions by the Bowyer algorithm until the characteristic size of
each simplex $\leq$ characteristic length field evaluated at the center of
its circumscribed circle/sphere.
\item
mesh post-processing and quality enforcement by node relocation and edge
swapping
\end{enumerate}
\end{slideitemize}

\end{slide}

% ---------------------------------------------------------------------------

\chapter{Solver integration}

\begin{slide}

\begin{minipage}{0.45\textwidth}
\emph{Batch}:
\begin{slideitemize}
\item Advanced users
\item No graphical overhead
\item Easily scriptable
\end{slideitemize}
For complex problems
\end{minipage}%
\begin{minipage}{0.1\textwidth}
vs.
\end{minipage}%
\begin{minipage}{0.45\textwidth}
\emph{Interactive}:
\begin{slideitemize}
\item Smoother learning curve
\item Better integration
\item Faster for testing
\end{slideitemize}
For testing/learning
\end{minipage}%

\bigskip\bigskip
\begin{center}
Solvers should provide \emph{both}
\end{center}

\end{slide}

\begin{slide}

Example: integration of GetDP
\begin{slideitemize}
\item \emph{High level (system calls)}: commercial applications (Matlab,
Mathematica, ...), shell scripts (bash, Perl, Python, ...), other
programming languages (C, C++, ...)
\item \emph{Medium level (sockets)}: customizable applications (Gmsh, ...),
shell scripts (Perl, Python, ...), other programming languages (C, C++, ...)
\item \emph{Low level (source code)}: programming languages (C, C++, ...)
\end{slideitemize}

\end{slide}

% ---------------------------------------------------------------------------

\chapter{Post-processing}

\begin{slide}

\begin{slideitemize}
\item \emph{Multiple views}, manipulated globally or individually
\item \emph{Scalar} views displayed by iso-value curves or color maps
\item \emph{Vector} and \emph{tensor} views displayed with 3D arrows or
displacement maps
\item Interactive \emph{functions} include offsets, elevation, interactive
color map modification, range clamping, interactive animation, etc.
\item All functions are \emph{scriptable} (animation, movement, ...)
\item Various \emph{output formats} (bitmap or vector)
\item \emph{Modular} (plug-in mechanism)
\end{slideitemize}

\end{slide}

% ---------------------------------------------------------------------------

